\chapter{Introduction}

The objective of the GridPACK$\mathrm{{}^{TM}}$ toolkit project is to provide a framework to support the rapid development of power grid applications capable of running on high performance computing architectures (HPC) with high levels of performance and scalability. The toolkit allows power system engineers to focus on developing working applications from their models without getting bogged down in the details of decomposing the computation across multiple processors, managing data transfers between processors, working out index transformations between power grid networks and the matrices generated by different power applications, and managing input and output. GridPACK is being designed to encapsulate as much of the book-keeping required to set up HPC applications as possible using high-level programming abstractions that allow developers to concentrate on the physics and mathematics of their problems.
This document summarizes the overall design of the GridPACK framework and provides a detailed description of its components. The remainder of this document will describe the functionality incorporated into the GridPACK framework to support multiple power grid applications. The framework will continue to evolve as more real-world experience can be incorporated into the design process but many base classes that have already been identified that are capable of supporting a range of applications.
During the initial stages of GridPACK development, four power grid applications were targeted for implementation. These included:
\begin{enumerate}
\item  Powerflow simulations of the electric grid

\item  Contingency analysis of the electric grid

\item  State estimation based on electric grid measurements

\item  Dynamic simulations of the electric grid
\end{enumerate}

From these applications, several cross-cutting functionalities were identified that could be used to support multiple applications.
\begin{enumerate}
\item  Network topology and behavior. The network topology is the starting point for any power grid analysis. The topology defines the initial network model and is the connection point between the physical problem definition in terms of buses and branches and the solution method, which is usually expressed in terms of matrices and vectors.

\item  Network components and their properties (e.g. bus and branch models, measurements, etc.). Grid components are the objects associated with the buses and branches of the power grid network. Along with the network topology itself, these define the physical system being modeled and in some cases the analysis that is to be performed. Bus and branch components can be differentiated into things like generators, loads, grounds, lines, transformers, measurements, etc. and depending on the how they are defined and the level of detail incorporated into them, they define different power grid systems and analyses. The behavior of buses and branches can depend on the properties of branches or buses that are directly attached to them, e.g. figuring out the contribution of a particular bus to the solution procedure may require that properties of the attached branches are made available to the bus. The necessity for exchanging this data is built into the framework. Furthermore, these data exchanges must also be accounted for in a parallel computing context, since the grid component from which data is required may be located on a different processor.

\item  Linear algebra and solvers. Basic algebraic objects, such as distributed matrices and vectors, are a core part of the solution algorithms required by power grid analyses. Most solution algorithms are dominated by sparse matrices but a few, such as Kalman filter analyses, require dense matrices. Vectors are typically dense. There exists a rich set of libraries for constructing distributed matrices and vectors and these are coupled to preconditioner and solver libraries. GridPACK can leverage this work heavily by creating wrappers to these libraries that can be used in solution algorithms. Wrapping these libraries instead of using them directly will have the advantage that creating algebraic objects can be simplified somewhat for power grid applications but more importantly, it will allow framework developers to investigate new solver and algebraic libraries seamlessly, without disrupting other parts of the code.

\item  Mapping between network and algebraic objects. The physical properties of power grid systems are defined by networks and the properties of the network components but the equations describing the networks are algebraic in nature. The mappings between the physical networks and the algebraic equations depend on the indexing scheme used to describe the network and the number of parameters in the network components that appear in the equations. Constructing a map between network parameters and their corresponding locations in a matrix or vector can be complicated and error prone. Fortunately, much of this work can be automated and developers can focus more on developing code to evaluate individual matrix elements without worrying about where to locate them in the matrix. This can simplify coding considerably.
\end{enumerate}

The elements described above have all been incorporated into GridPACK modules. More details about these modules and their interactions are provided in the remainder of this document.
