\documentclass[12pt]{article}

\usepackage{listings}
\lstset{
    language = C++, % language is C++                                                                                     
    frame=tb, % draw a frame at the top and bottom of the code block                                                      
    tabsize=4, % tab space width                                                                                          
    showstringspaces=false, % don't mark spaces in strings                                                                
  %  numbers=left, % display line numbers on the left                                                                     
    commentstyle=\color{red}, % comment color                                                                             
    keywordstyle=\color{blue}, % keyword color                                                                            
    stringstyle=\color{red} % string color                                                                                
}
\usepackage{xcolor}
\usepackage[toc,page]{appendix}
\usepackage{amsmath}
\usepackage{amsfonts}
\usepackage{amssymb}
\usepackage{mathtools}
\usepackage{verbatim}
\usepackage{url}

\newcommand{\dw}{\Delta\omega}
\newcommand{\Xdp}{X{'}_d}
\newcommand{\Ep}{E^{'}}
\newcommand{\Vref}{V_{REF}}

\begin{document}

% --------------------------------------------------------------                                                          
%                         Start here                                                                                      
% --------------------------------------------------------------                                                          

\title{Dynamics Models implemented in GridPACK}

\maketitle

\section{Introduction}
This document describes the variables, equations, inputs, and outputs for the different generator, exciter, and turbine governor models implemented in GridPACK.
The document assumes the bus voltage is expressed in cartesian form, i.e. $\bar{V} = V_R + iV_I$, where
$V_R$ and $V_I$ are the real and imaginary components of the complex bus voltage $V$.

\section{Generator models}
The inputs and outputs for the generator models are given below. These inputs/outputs representing
the coupling with the bus and the generator controls such as exciters, turbine-governors.
\subsection{Inputs}
\begin{enumerate}
  \item $P_M$: Mechanical power input
\end{enumerate}

\subsection{Outputs}
\begin{enumerate}
  \item Generator current real and imaginary components, $I_{GR}$ and $I_{GI}$, respectively.
  \item $\delta$ : Rotor angle
  \item $\dw$ : Speed deviation
  \item Field current $I_{FD}$
\end{enumerate}

\subsection{GENCLS\cite{GENCLS}}
Note: In the GridPACK implementation, the speed effects mimicing governor reaction is ignored.

\subsubsection{Variables}
\begin{enumerate}
  \item $\delta$: Machine angle
  \item $\dw$: Machine speed deviation (per unit)
\end{enumerate}

\subsubsection{Equations}
\begin{align}
\dfrac{1}{\omega_s}\dfrac{d\delta}{dt} &= \dw \\
2H\dfrac{d\dw}{dt} &= P_M - \dfrac{V_R\Ep\sin(\delta) + V_I\Ep\cos(\delta)}{\Xdp} - D\dw
\end{align}

\subsubsection{Outputs}
\begin{enumerate}
	\item Generator currents:
\begin{align}
  I_{GR} &= \dfrac{\Ep\sin(\delta) - V_I}{\Xdp} \\
  I_{GI} &= \dfrac{-\Ep\cos(\delta) + V_R}{\Xdp}
\end{align}
\end{enumerate}

\subsection{GENROU}

\subsection{GENSAL}

\section{Exciter models}
Over/under excitation limiters and stabilizers are not implemented in GridPACK. As a result, any inputs from these devices are not considered in the exciter models.

\subsection{Inputs}
\begin{enumerate}
  \item $E_c$: Terminal voltage magnitude = $\sqrt{V^{2}_R + V^{2}_I}$
  \item $\Vref$: Reference voltage
  \item $I_{LR}$: Field current limiter parameter
\end{enumerate}

\subsection{Outputs}
\begin{enumerate}
  \item $E_{FD}$: Field voltage
\end{enumerate}

\subsection{EXDC1}

\subsection{ESST1A \cite{ESST1A}}

\subsubsection{Variables}
\begin{enumerate}
  \item $V_{meas}$: Measured terminal voltage
  \item $x_{LL1}$: First lead lag block state variable
  \item $x_{LL2}$: Second lead lag block state variable
  \item $V_A$: Gain voltage
  \item $x_F$: Feedback loop state variable
\end{enumerate}

\subsubsection{Equations}
\begin{flalign}
  &\begin{cases}
    T_R\dfrac{dV_{meas}}{dt} = Ec - V_{meas},&\text{if}~~T_R > 0 \\
    0 = Ec - V_{meas},&\text{if}~~T_R = 0
  \end{cases}& \\
  &\begin{cases}
    T_b\dfrac{dx_{LL1}}{dt} = (1 - \dfrac{T_c}{T_b})V_{in} - x_{LL1},&\text{if}~~T_b > 0 \\
    0 = V_{in} - x_{LL1},&\text{if}~~T_b = 0
  \end{cases}& \\
  &\begin{cases}
    T_{b1}\dfrac{dx_{LL2}}{dt} = (1 - \dfrac{T_{c1}}{T_{b1}})y_{LL1} - x_{LL2},&\text{if}~~T_{b1} > 0 \\
    0 = y_{LL1} - x_{LL2},&\text{if}~~T_{b1} = 0
  \end{cases}& \\
  &\begin{cases}
    T_{A}\dfrac{dV_{A}}{dt} = y_{LL2} -K_AV_A,&\text{if}~~T_{A} > 0,\quad V_{AMIN} \le V_A \le V_{AMAX} \\
    0 = y_{LL2} - V_{A},&\text{if}~~T_{A} = 0,\quad V_{AMIN} \le V_A \le V_{AMAX}
  \end{cases}& \\
  &T_F\dfrac{dx_F}{dt} = \dfrac{K_Fu_{FB}}{T_F} - x_F&
\end{flalign}
Here, the intermediate variables used in the calculations are as follows:
\begin{align*}
  &V_{in} = \min(\max(\Vref - V_{meas} - V_F,V_{IMIN}),V_{IMAX})& \\
  &\begin{cases}
    y_{LL1} = x_{LL1} + \dfrac{T_c}{T_b}V_{in},&\text{if}~~T_{b} > 0 \\
    y_{LL1} = x_{LL1},&\text{if}~~T_{b} = 0
  \end{cases}& \\
  &\begin{cases}
    y_{LL2} = x_{LL2} + \dfrac{T_{c1}}{T_{b1}}y_{LL1},&\text{if}~~T_{b1} > 0 \\
    y_{LL2} = x_{LL2},&\text{if}~~T_{b1} = 0 \\
  \end{cases}& \\
  &u_{FB}  = V_A - \min(K_{LR}(I_{FD} - I_{LR}),0)& \\
  &V_F     = x_F + \dfrac{K_FT_F}{u_{FB}}&   
\end{align*}

\subsection{Outputs}
\begin{enumerate}
  \item $E_{FD} = \min(\max(u_{FB},V_TV_{RMIN}),V_TV_{RMAX} - K_CI_{FD})$
\end{enumerate}

\section{Turbine governor models}

\subsection{WSIEG1}

\begin{thebibliography}{20}

\bibitem{GENCLS}
  PowerWorld,
  \textit{GENCLS model block diagram}
  available at \url{https://www.powerworld.com/WebHelp/Content/TransientModels_HTML/Machine\%20Mode\%20GENCLS.htm}


\bibitem{ESST1A}
  PowerWorld,
  \textit{ESST1A model block diagram}
  available at \url{https://www.powerworld.com/WebHelp/Content/TransientModels_HTML/Exciter\%20ESST1A\%20and\%20ESST1A_GE.htm},

\end{thebibliography}

\end{document}
