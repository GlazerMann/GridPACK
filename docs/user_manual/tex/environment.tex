\section{Environment}\label{environment}

GridPACK applications need to initialize several libraries in order to execute
properly. These can be initialized explicitly by the user in their
application but they can also be initialized by creating a single
\texttt{\textbf{Environment}} object at the start of the code. The constructor
for this object will automatically call all the appropriate initialization
functions for libraries used by GridPACK. This object can also be used to
support an inline help message that can be used to document how to use the
application.

There are two main constructors for this class

{
\color{red}
\begin{Verbatim}[fontseries=b]
Environment(int argc, char **argv)
\end{Verbatim}
}

{
\color{red}
\begin{Verbatim}[fontseries=b]
Environment(int argc, char **argv, const char* help)
\end{Verbatim}
}

The \texttt{\textbf{argc}} and \texttt{\textbf{argv}} arguments are the standard
command line variables used in C and C++ \texttt{\textbf{main}} programs. These
will be passed to the math library and MPI initialization. Other options can also be
passed from the command line as well. Currently, the only command line
options that are supported directly by the \texttt{\textbf{Environment}} class
are \texttt{\textbf{-h}} and \texttt{\textbf{-help}}. If the application is invoked
with these options, then the program will print out whatever information is stored
in the \texttt{\textbf{help}} variable and then exit.
