\section{Advanced Network Functionality}

Most users will create networks be reading in an external configuration file using a parser and let that function create and populate the network. Users that are interested in creating networks through another channel will need functionality in the \texttt{\textbf{BaseNetwork}} class that was not described earlier. These can be used, for example, to write routines for creating networks from external configuration formats not currently supported in GridPACK. The functions described below can be used to populate a network directly with buses and branches.

When a network is originally created, it is just an empty object with no buses or branches associated with it. Adding buses and branches directly to a networks is straightforward in certain respects and complicated in others. Buses and branches can be added to the network in any order and any process can add buses and branches without regard to topology. The partition algorithm will sort out which buses and branches should be grouped together based on connectivity, as well as adding whatever ghost buses and branches that are required to the system. Buses are originally characterized by a unique index. This index does not need to start at 0 or 1 and the indices do not need to form a contiguous sequence. The only requirement is that each bus has a unique index. When using one of the GridPACK parsers to read in an external configuration file, GridPACK internally assigns a second index, called the global index, that starts at 0 and forms a continuous sequence that runs up to N-1, where N is the number of unique buses in the network. Information written out by the serial IO modules, for example, will be ordered based on the global index. To add a bus to the network and to get all other module functions to work, the user must assign both these indices to the bus. The function to add a new bus to the network is

{
\color{red}
\begin{Verbatim}[fontseries=b]
void addBus(int idx);
\end{Verbatim}
}

The argument \texttt{\textbf{idx}} is the original index of the bus. The function to set the global index of the bus is

{
\color{red}
\begin{Verbatim}[fontseries=b]
bool setGlobalBusIndex(int idx, int gdx);
\end{Verbatim}
}

The argument \texttt{\textbf{idx}} is the local index of the bus in the network, the argument \texttt{\textbf{gdx}} is the global index assigned by the user. This function will return false if the local index exceeds the number of buses on the processor. The existing GridPACK parsers assign the original index based on the index used in the configuration file and the global index based on the position of the bus in the configuration file. For other sources, it may be necessary for users to develop their own strategies for assigning indices.

Once the bus has been added, its properties can be modified using methods in the \texttt{\textbf{BaseBusComponent}} class. Note that creating the bus simultaneously creates the associated \texttt{\textbf{DataCollection}} object. This can be accessed using the network function

{
\color{red}
\begin{Verbatim}[fontseries=b]
boost::shared_ptr<DataCollection> getBusData(int idx);
\end{Verbatim}
}

where \texttt{\textbf{idx}} is once again the local bus index. Once the data collection object is available, properties can be added to it as described earlier.

New buses are added to the system with the assumption that they are local to the processor. This means the ``active'' flag is originally set to true. The partitioner can then be used to redistribute buses and branches and add ghost components. If the user wants to set their own local/ghost status, then this can be done through the function

{
\color{red}
\begin{Verbatim}[fontseries=b]
bool setActiveBus(int idx, bool flag);
\end{Verbatim}
}

The index \texttt{\textbf{idx}} is a local index, \texttt{\textbf{flag}} is the status of the bus (true for local buses, false for ghost buses) and the function returns false if the local bus index does not correspond to a bus on the process.
The status of a bus as a reference bus can be set using the function

{
\color{red}
\begin{Verbatim}[fontseries=b]
void setReferenceBus(int idx);
\end{Verbatim}
}

where \texttt{\textbf{idx}} is a local index.By default, buses are created as ordinary buses.
Branches can be added to the system using a similar set of functions. Note that there is no requirement that branches be created on processes that contain either of the endpoint buses. In an extreme case, the complete set of buses and branches can be created on separate processes. To add a new branch to the system, the user must supply the original indices of the buses at each end of the branch and a global index for each individual branch. Again, the global index runs consecutively from 0 to M-1, where M is the total number of unique branches in the system. A new branch is added to the network using the function

{
\color{red}
\begin{Verbatim}[fontseries=b]
void addBranch(int idx1, int idx2);
\end{Verbatim}
}

where \texttt{\textbf{idx1}} is the original index of the ``from'' bus and \texttt{\textbf{idx2}} is the original index of the ``to'' bus. The global index of the branch can be set with the function

{
\color{red}
\begin{Verbatim}[fontseries=b]
bool setGlobalBranchIndex(int idx, int gdx);
\end{Verbatim}
}

where \texttt{\textbf{idx}} is the local index of the branch on the processor and \texttt{\textbf{gdx}} is the global index. As in the case of buses, the complicated part of adding a branch to the network is to determine a global index for the branch.
When a branch is created, a \texttt{\textbf{DateCollection}} object for the branch is created along and can be accessed using

{
\color{red}
\begin{Verbatim}[fontseries=b]
boost::shared_ptr<DataCollection> getBranchData(int idx);
\end{Verbatim}
}

where \texttt{\textbf{idx}} is the local index of the branch. Once a pointer to the data collection object is available, parameters can be added to it or modified as described earlier. The active status of the branch can be set with

{
\color{red}
\begin{Verbatim}[fontseries=b]
bool setActiveBranch(int idx, bool flag);
\end{Verbatim}
}

The arguments \texttt{\textbf{idx}} and \texttt{\textbf{flag}} are the local branch index and the branch status and the function returns false if the local index is not in the range of branches on the processor.

These functions are all that is needed to create a network from scratch or to write a parser for a new network configuration file format. These are currently used in the PSS/E parser classes to implement the network setup functionality.
