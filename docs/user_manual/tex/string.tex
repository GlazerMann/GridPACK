\section{String Utilities}

At some point, users may want to develop their own parsers for reading in information in external files. The \texttt{\textbf{StringUtils}} class is contained in the \texttt{\textbf{gridpack::utility}} namespace and is designed to provide some useful string manipulation routines that can be used to parse individual lines of a file. Other capabilities are available in standard C routines such as \texttt{\textbf{strcmp}} and the Boost libraries also have many useful routines. The \texttt{\textbf{StringUtils}} class is just a convenient container for different string manipulation methods; it has no internal state.

Some basic routines for modifying strings so that they can be compared with other strings are

{
\color{red}
\begin{Verbatim}[fontseries=b]
void trim(std::string &str)
\end{Verbatim}
}

which can be used to remove white space at the beginning and end of a string. This function will also convert all tabs and carriage returns to white space before trimming the white space at the ends of the string. The functions

{
\color{red}
\begin{Verbatim}[fontseries=b]
void toUpper(std::string &str)

void toLower(std::string &str)
\end{Verbatim}
}

can be used to convert all characters in the string to either upper or lower case.

Many devices in power grid applications are characterized by a one or two character alphanumeric string. It is useful to get these strings into a standard form so that they can be compared with other strings. The function

{
\color{red}
\begin{Verbatim}[fontseries=b]
std::string clean2Char(std::string &str)
\end{Verbatim}
}

returns a two character string that is right justified. It will also remove any quotes that may or may not be around the original string. The strings C1, `C1', ``C1'' and ``  C1'' will all return a string containing the two characters C1. A single character string will return a two character string with a blank as the first character.

The function

{
\color{red}
\begin{Verbatim}[fontseries=b]
std:: string trimQuotes(std::string &string)
\end{Verbatim}
}

can be used to remove either single or double quotation marks from around a string and remove any remaining white space at the beginning and end of the string.

Tokenizers can be used to break up a long string into individual elements. This
is useful for comma or blank delimitted strings representing a sequence of
values. The function

{
\color{red}
\begin{Verbatim}[fontseries=b]
std::vector<std::string> blankTokenizer(std::string &str)
\end{Verbatim}
}

will take a string in which individual elements are delimited by blank spaces and return a vector in which each element is a separate string (token). This function treats anything inside the original string that may be delimited by quotes as a single token, even if there are additional blank spaces between the quotes. Thus, the string

{
\color{red}
\begin{Verbatim}[fontseries=b]
1 5 "ATLANTA 001" 0.00056 1.02
\end{Verbatim}
}

is broken up into a vector containing the strings

{
\color{red}
\begin{Verbatim}[fontseries=b]
1

5

"ATLANTA 001"

0.00056

1.02
\end{Verbatim}
}

Both single and double quotes can be used as delimiters for internal strings.

The function
{
\color{red}
\begin{Verbatim}[fontseries=b]
std::vector<std::string> charTokenizer(std::string &str, const char *sep)
\end{Verbatim}
}

will break up a string using the character string \texttt{\textbf{sep}} as a delimiter.
Thus, the string

{
\color{red}
\begin{Verbatim}[fontseries=b]
1, 5, "ATLANTA 001", 0.00056, 1.02
\end{Verbatim}
}

would be broken up into a vector containing the strings

{
\color{red}
\begin{Verbatim}[fontseries=b]
1

 5

 "ATLANTA 001"

 0.00056

 1.02
\end{Verbatim}
}
Note the presence of the leading white space in the last four tokens. This may
need to be removed using the \texttt{\textbf{trim}} function before using these
strings in any comparisons.

Finally, the functions

{
\color{red}
\begin{Verbatim}[fontseries=b]
bool getBool(std::string &str)
bool getBool(str)
\end{Verbatim}
}

can be used to convert a string to a bool value. These functions
will convert the strings ``True'', ``Yes'', ``T'', ``Y'' and ``1'' to
the bool value ``true'' and the strings ``False'', ``No'', ``F'', ``N'' and
``0'' to the bool value ``false''. Both functions are case insensitive and will
treat the strings ``TRUE'', ``True'' and ``true'', etc. as equivalent.
