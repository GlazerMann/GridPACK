\chapter{Application Modules}

Many of the example applications in GridPACK have been converted to modules that can be called from other programs. These modules make it relatively simple to chain different types of calculations together to form larger applications. An example is using power flow or state estimation to initialize a dynamic simulation. The modules are designed to separat the major phases of the calculation into calls so that users have some fine-grained control that allows them to mix different applications together. In most cases, different options for setting up calculations are provided so that once a network has been read in and partitioned, it is not necessary to repeat this process when a new calculation is started based on the results of a previous simulation.

Currently, three applications are available as modules within GridPACK. They include power flow, state estimation, and dynamic simulation using the full Y-matrix. Each of these modules can be used to create a short, standalone application, but the goal is to enable users to combine modules together in more complicated work flows. These modules can also be used as a starting point for users to create their own applications by modifying the existing code in the modules to create new functionality. Each of the modules is described in more detail below. Example codes that use the modules to implement applications can be found in the \texttt{\textbf{src/application}} directory. These include powerflow, state estimation, contingency analysis and dynamic simulation. These directories also contain sample input networks and input files. Options for different solvers can be found in these files.
